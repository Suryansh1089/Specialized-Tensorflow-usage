
% Default to the notebook output style

    


% Inherit from the specified cell style.




    
\documentclass[11pt]{article}

    
    
    \usepackage[T1]{fontenc}
    % Nicer default font (+ math font) than Computer Modern for most use cases
    \usepackage{mathpazo}

    % Basic figure setup, for now with no caption control since it's done
    % automatically by Pandoc (which extracts ![](path) syntax from Markdown).
    \usepackage{graphicx}
    % We will generate all images so they have a width \maxwidth. This means
    % that they will get their normal width if they fit onto the page, but
    % are scaled down if they would overflow the margins.
    \makeatletter
    \def\maxwidth{\ifdim\Gin@nat@width>\linewidth\linewidth
    \else\Gin@nat@width\fi}
    \makeatother
    \let\Oldincludegraphics\includegraphics
    % Set max figure width to be 80% of text width, for now hardcoded.
    \renewcommand{\includegraphics}[1]{\Oldincludegraphics[width=.8\maxwidth]{#1}}
    % Ensure that by default, figures have no caption (until we provide a
    % proper Figure object with a Caption API and a way to capture that
    % in the conversion process - todo).
    \usepackage{caption}
    \DeclareCaptionLabelFormat{nolabel}{}
    \captionsetup{labelformat=nolabel}

    \usepackage{adjustbox} % Used to constrain images to a maximum size 
    \usepackage{xcolor} % Allow colors to be defined
    \usepackage{enumerate} % Needed for markdown enumerations to work
    \usepackage{geometry} % Used to adjust the document margins
    \usepackage{amsmath} % Equations
    \usepackage{amssymb} % Equations
    \usepackage{textcomp} % defines textquotesingle
    % Hack from http://tex.stackexchange.com/a/47451/13684:
    \AtBeginDocument{%
        \def\PYZsq{\textquotesingle}% Upright quotes in Pygmentized code
    }
    \usepackage{upquote} % Upright quotes for verbatim code
    \usepackage{eurosym} % defines \euro
    \usepackage[mathletters]{ucs} % Extended unicode (utf-8) support
    \usepackage[utf8x]{inputenc} % Allow utf-8 characters in the tex document
    \usepackage{fancyvrb} % verbatim replacement that allows latex
    \usepackage{grffile} % extends the file name processing of package graphics 
                         % to support a larger range 
    % The hyperref package gives us a pdf with properly built
    % internal navigation ('pdf bookmarks' for the table of contents,
    % internal cross-reference links, web links for URLs, etc.)
    \usepackage{hyperref}
    \usepackage{longtable} % longtable support required by pandoc >1.10
    \usepackage{booktabs}  % table support for pandoc > 1.12.2
    \usepackage[inline]{enumitem} % IRkernel/repr support (it uses the enumerate* environment)
    \usepackage[normalem]{ulem} % ulem is needed to support strikethroughs (\sout)
                                % normalem makes italics be italics, not underlines
    \usepackage{mathrsfs}
    

    
    
    % Colors for the hyperref package
    \definecolor{urlcolor}{rgb}{0,.145,.698}
    \definecolor{linkcolor}{rgb}{.71,0.21,0.01}
    \definecolor{citecolor}{rgb}{.12,.54,.11}

    % ANSI colors
    \definecolor{ansi-black}{HTML}{3E424D}
    \definecolor{ansi-black-intense}{HTML}{282C36}
    \definecolor{ansi-red}{HTML}{E75C58}
    \definecolor{ansi-red-intense}{HTML}{B22B31}
    \definecolor{ansi-green}{HTML}{00A250}
    \definecolor{ansi-green-intense}{HTML}{007427}
    \definecolor{ansi-yellow}{HTML}{DDB62B}
    \definecolor{ansi-yellow-intense}{HTML}{B27D12}
    \definecolor{ansi-blue}{HTML}{208FFB}
    \definecolor{ansi-blue-intense}{HTML}{0065CA}
    \definecolor{ansi-magenta}{HTML}{D160C4}
    \definecolor{ansi-magenta-intense}{HTML}{A03196}
    \definecolor{ansi-cyan}{HTML}{60C6C8}
    \definecolor{ansi-cyan-intense}{HTML}{258F8F}
    \definecolor{ansi-white}{HTML}{C5C1B4}
    \definecolor{ansi-white-intense}{HTML}{A1A6B2}
    \definecolor{ansi-default-inverse-fg}{HTML}{FFFFFF}
    \definecolor{ansi-default-inverse-bg}{HTML}{000000}

    % commands and environments needed by pandoc snippets
    % extracted from the output of `pandoc -s`
    \providecommand{\tightlist}{%
      \setlength{\itemsep}{0pt}\setlength{\parskip}{0pt}}
    \DefineVerbatimEnvironment{Highlighting}{Verbatim}{commandchars=\\\{\}}
    % Add ',fontsize=\small' for more characters per line
    \newenvironment{Shaded}{}{}
    \newcommand{\KeywordTok}[1]{\textcolor[rgb]{0.00,0.44,0.13}{\textbf{{#1}}}}
    \newcommand{\DataTypeTok}[1]{\textcolor[rgb]{0.56,0.13,0.00}{{#1}}}
    \newcommand{\DecValTok}[1]{\textcolor[rgb]{0.25,0.63,0.44}{{#1}}}
    \newcommand{\BaseNTok}[1]{\textcolor[rgb]{0.25,0.63,0.44}{{#1}}}
    \newcommand{\FloatTok}[1]{\textcolor[rgb]{0.25,0.63,0.44}{{#1}}}
    \newcommand{\CharTok}[1]{\textcolor[rgb]{0.25,0.44,0.63}{{#1}}}
    \newcommand{\StringTok}[1]{\textcolor[rgb]{0.25,0.44,0.63}{{#1}}}
    \newcommand{\CommentTok}[1]{\textcolor[rgb]{0.38,0.63,0.69}{\textit{{#1}}}}
    \newcommand{\OtherTok}[1]{\textcolor[rgb]{0.00,0.44,0.13}{{#1}}}
    \newcommand{\AlertTok}[1]{\textcolor[rgb]{1.00,0.00,0.00}{\textbf{{#1}}}}
    \newcommand{\FunctionTok}[1]{\textcolor[rgb]{0.02,0.16,0.49}{{#1}}}
    \newcommand{\RegionMarkerTok}[1]{{#1}}
    \newcommand{\ErrorTok}[1]{\textcolor[rgb]{1.00,0.00,0.00}{\textbf{{#1}}}}
    \newcommand{\NormalTok}[1]{{#1}}
    
    % Additional commands for more recent versions of Pandoc
    \newcommand{\ConstantTok}[1]{\textcolor[rgb]{0.53,0.00,0.00}{{#1}}}
    \newcommand{\SpecialCharTok}[1]{\textcolor[rgb]{0.25,0.44,0.63}{{#1}}}
    \newcommand{\VerbatimStringTok}[1]{\textcolor[rgb]{0.25,0.44,0.63}{{#1}}}
    \newcommand{\SpecialStringTok}[1]{\textcolor[rgb]{0.73,0.40,0.53}{{#1}}}
    \newcommand{\ImportTok}[1]{{#1}}
    \newcommand{\DocumentationTok}[1]{\textcolor[rgb]{0.73,0.13,0.13}{\textit{{#1}}}}
    \newcommand{\AnnotationTok}[1]{\textcolor[rgb]{0.38,0.63,0.69}{\textbf{\textit{{#1}}}}}
    \newcommand{\CommentVarTok}[1]{\textcolor[rgb]{0.38,0.63,0.69}{\textbf{\textit{{#1}}}}}
    \newcommand{\VariableTok}[1]{\textcolor[rgb]{0.10,0.09,0.49}{{#1}}}
    \newcommand{\ControlFlowTok}[1]{\textcolor[rgb]{0.00,0.44,0.13}{\textbf{{#1}}}}
    \newcommand{\OperatorTok}[1]{\textcolor[rgb]{0.40,0.40,0.40}{{#1}}}
    \newcommand{\BuiltInTok}[1]{{#1}}
    \newcommand{\ExtensionTok}[1]{{#1}}
    \newcommand{\PreprocessorTok}[1]{\textcolor[rgb]{0.74,0.48,0.00}{{#1}}}
    \newcommand{\AttributeTok}[1]{\textcolor[rgb]{0.49,0.56,0.16}{{#1}}}
    \newcommand{\InformationTok}[1]{\textcolor[rgb]{0.38,0.63,0.69}{\textbf{\textit{{#1}}}}}
    \newcommand{\WarningTok}[1]{\textcolor[rgb]{0.38,0.63,0.69}{\textbf{\textit{{#1}}}}}
    
    
    % Define a nice break command that doesn't care if a line doesn't already
    % exist.
    \def\br{\hspace*{\fill} \\* }
    % Math Jax compatibility definitions
    \def\gt{>}
    \def\lt{<}
    \let\Oldtex\TeX
    \let\Oldlatex\LaTeX
    \renewcommand{\TeX}{\textrm{\Oldtex}}
    \renewcommand{\LaTeX}{\textrm{\Oldlatex}}
    % Document parameters
    % Document title
    \title{Week 4 Programming Assignment}
    
    
    
    
    

    % Pygments definitions
    
\makeatletter
\def\PY@reset{\let\PY@it=\relax \let\PY@bf=\relax%
    \let\PY@ul=\relax \let\PY@tc=\relax%
    \let\PY@bc=\relax \let\PY@ff=\relax}
\def\PY@tok#1{\csname PY@tok@#1\endcsname}
\def\PY@toks#1+{\ifx\relax#1\empty\else%
    \PY@tok{#1}\expandafter\PY@toks\fi}
\def\PY@do#1{\PY@bc{\PY@tc{\PY@ul{%
    \PY@it{\PY@bf{\PY@ff{#1}}}}}}}
\def\PY#1#2{\PY@reset\PY@toks#1+\relax+\PY@do{#2}}

\expandafter\def\csname PY@tok@w\endcsname{\def\PY@tc##1{\textcolor[rgb]{0.73,0.73,0.73}{##1}}}
\expandafter\def\csname PY@tok@c\endcsname{\let\PY@it=\textit\def\PY@tc##1{\textcolor[rgb]{0.25,0.50,0.50}{##1}}}
\expandafter\def\csname PY@tok@cp\endcsname{\def\PY@tc##1{\textcolor[rgb]{0.74,0.48,0.00}{##1}}}
\expandafter\def\csname PY@tok@k\endcsname{\let\PY@bf=\textbf\def\PY@tc##1{\textcolor[rgb]{0.00,0.50,0.00}{##1}}}
\expandafter\def\csname PY@tok@kp\endcsname{\def\PY@tc##1{\textcolor[rgb]{0.00,0.50,0.00}{##1}}}
\expandafter\def\csname PY@tok@kt\endcsname{\def\PY@tc##1{\textcolor[rgb]{0.69,0.00,0.25}{##1}}}
\expandafter\def\csname PY@tok@o\endcsname{\def\PY@tc##1{\textcolor[rgb]{0.40,0.40,0.40}{##1}}}
\expandafter\def\csname PY@tok@ow\endcsname{\let\PY@bf=\textbf\def\PY@tc##1{\textcolor[rgb]{0.67,0.13,1.00}{##1}}}
\expandafter\def\csname PY@tok@nb\endcsname{\def\PY@tc##1{\textcolor[rgb]{0.00,0.50,0.00}{##1}}}
\expandafter\def\csname PY@tok@nf\endcsname{\def\PY@tc##1{\textcolor[rgb]{0.00,0.00,1.00}{##1}}}
\expandafter\def\csname PY@tok@nc\endcsname{\let\PY@bf=\textbf\def\PY@tc##1{\textcolor[rgb]{0.00,0.00,1.00}{##1}}}
\expandafter\def\csname PY@tok@nn\endcsname{\let\PY@bf=\textbf\def\PY@tc##1{\textcolor[rgb]{0.00,0.00,1.00}{##1}}}
\expandafter\def\csname PY@tok@ne\endcsname{\let\PY@bf=\textbf\def\PY@tc##1{\textcolor[rgb]{0.82,0.25,0.23}{##1}}}
\expandafter\def\csname PY@tok@nv\endcsname{\def\PY@tc##1{\textcolor[rgb]{0.10,0.09,0.49}{##1}}}
\expandafter\def\csname PY@tok@no\endcsname{\def\PY@tc##1{\textcolor[rgb]{0.53,0.00,0.00}{##1}}}
\expandafter\def\csname PY@tok@nl\endcsname{\def\PY@tc##1{\textcolor[rgb]{0.63,0.63,0.00}{##1}}}
\expandafter\def\csname PY@tok@ni\endcsname{\let\PY@bf=\textbf\def\PY@tc##1{\textcolor[rgb]{0.60,0.60,0.60}{##1}}}
\expandafter\def\csname PY@tok@na\endcsname{\def\PY@tc##1{\textcolor[rgb]{0.49,0.56,0.16}{##1}}}
\expandafter\def\csname PY@tok@nt\endcsname{\let\PY@bf=\textbf\def\PY@tc##1{\textcolor[rgb]{0.00,0.50,0.00}{##1}}}
\expandafter\def\csname PY@tok@nd\endcsname{\def\PY@tc##1{\textcolor[rgb]{0.67,0.13,1.00}{##1}}}
\expandafter\def\csname PY@tok@s\endcsname{\def\PY@tc##1{\textcolor[rgb]{0.73,0.13,0.13}{##1}}}
\expandafter\def\csname PY@tok@sd\endcsname{\let\PY@it=\textit\def\PY@tc##1{\textcolor[rgb]{0.73,0.13,0.13}{##1}}}
\expandafter\def\csname PY@tok@si\endcsname{\let\PY@bf=\textbf\def\PY@tc##1{\textcolor[rgb]{0.73,0.40,0.53}{##1}}}
\expandafter\def\csname PY@tok@se\endcsname{\let\PY@bf=\textbf\def\PY@tc##1{\textcolor[rgb]{0.73,0.40,0.13}{##1}}}
\expandafter\def\csname PY@tok@sr\endcsname{\def\PY@tc##1{\textcolor[rgb]{0.73,0.40,0.53}{##1}}}
\expandafter\def\csname PY@tok@ss\endcsname{\def\PY@tc##1{\textcolor[rgb]{0.10,0.09,0.49}{##1}}}
\expandafter\def\csname PY@tok@sx\endcsname{\def\PY@tc##1{\textcolor[rgb]{0.00,0.50,0.00}{##1}}}
\expandafter\def\csname PY@tok@m\endcsname{\def\PY@tc##1{\textcolor[rgb]{0.40,0.40,0.40}{##1}}}
\expandafter\def\csname PY@tok@gh\endcsname{\let\PY@bf=\textbf\def\PY@tc##1{\textcolor[rgb]{0.00,0.00,0.50}{##1}}}
\expandafter\def\csname PY@tok@gu\endcsname{\let\PY@bf=\textbf\def\PY@tc##1{\textcolor[rgb]{0.50,0.00,0.50}{##1}}}
\expandafter\def\csname PY@tok@gd\endcsname{\def\PY@tc##1{\textcolor[rgb]{0.63,0.00,0.00}{##1}}}
\expandafter\def\csname PY@tok@gi\endcsname{\def\PY@tc##1{\textcolor[rgb]{0.00,0.63,0.00}{##1}}}
\expandafter\def\csname PY@tok@gr\endcsname{\def\PY@tc##1{\textcolor[rgb]{1.00,0.00,0.00}{##1}}}
\expandafter\def\csname PY@tok@ge\endcsname{\let\PY@it=\textit}
\expandafter\def\csname PY@tok@gs\endcsname{\let\PY@bf=\textbf}
\expandafter\def\csname PY@tok@gp\endcsname{\let\PY@bf=\textbf\def\PY@tc##1{\textcolor[rgb]{0.00,0.00,0.50}{##1}}}
\expandafter\def\csname PY@tok@go\endcsname{\def\PY@tc##1{\textcolor[rgb]{0.53,0.53,0.53}{##1}}}
\expandafter\def\csname PY@tok@gt\endcsname{\def\PY@tc##1{\textcolor[rgb]{0.00,0.27,0.87}{##1}}}
\expandafter\def\csname PY@tok@err\endcsname{\def\PY@bc##1{\setlength{\fboxsep}{0pt}\fcolorbox[rgb]{1.00,0.00,0.00}{1,1,1}{\strut ##1}}}
\expandafter\def\csname PY@tok@kc\endcsname{\let\PY@bf=\textbf\def\PY@tc##1{\textcolor[rgb]{0.00,0.50,0.00}{##1}}}
\expandafter\def\csname PY@tok@kd\endcsname{\let\PY@bf=\textbf\def\PY@tc##1{\textcolor[rgb]{0.00,0.50,0.00}{##1}}}
\expandafter\def\csname PY@tok@kn\endcsname{\let\PY@bf=\textbf\def\PY@tc##1{\textcolor[rgb]{0.00,0.50,0.00}{##1}}}
\expandafter\def\csname PY@tok@kr\endcsname{\let\PY@bf=\textbf\def\PY@tc##1{\textcolor[rgb]{0.00,0.50,0.00}{##1}}}
\expandafter\def\csname PY@tok@bp\endcsname{\def\PY@tc##1{\textcolor[rgb]{0.00,0.50,0.00}{##1}}}
\expandafter\def\csname PY@tok@fm\endcsname{\def\PY@tc##1{\textcolor[rgb]{0.00,0.00,1.00}{##1}}}
\expandafter\def\csname PY@tok@vc\endcsname{\def\PY@tc##1{\textcolor[rgb]{0.10,0.09,0.49}{##1}}}
\expandafter\def\csname PY@tok@vg\endcsname{\def\PY@tc##1{\textcolor[rgb]{0.10,0.09,0.49}{##1}}}
\expandafter\def\csname PY@tok@vi\endcsname{\def\PY@tc##1{\textcolor[rgb]{0.10,0.09,0.49}{##1}}}
\expandafter\def\csname PY@tok@vm\endcsname{\def\PY@tc##1{\textcolor[rgb]{0.10,0.09,0.49}{##1}}}
\expandafter\def\csname PY@tok@sa\endcsname{\def\PY@tc##1{\textcolor[rgb]{0.73,0.13,0.13}{##1}}}
\expandafter\def\csname PY@tok@sb\endcsname{\def\PY@tc##1{\textcolor[rgb]{0.73,0.13,0.13}{##1}}}
\expandafter\def\csname PY@tok@sc\endcsname{\def\PY@tc##1{\textcolor[rgb]{0.73,0.13,0.13}{##1}}}
\expandafter\def\csname PY@tok@dl\endcsname{\def\PY@tc##1{\textcolor[rgb]{0.73,0.13,0.13}{##1}}}
\expandafter\def\csname PY@tok@s2\endcsname{\def\PY@tc##1{\textcolor[rgb]{0.73,0.13,0.13}{##1}}}
\expandafter\def\csname PY@tok@sh\endcsname{\def\PY@tc##1{\textcolor[rgb]{0.73,0.13,0.13}{##1}}}
\expandafter\def\csname PY@tok@s1\endcsname{\def\PY@tc##1{\textcolor[rgb]{0.73,0.13,0.13}{##1}}}
\expandafter\def\csname PY@tok@mb\endcsname{\def\PY@tc##1{\textcolor[rgb]{0.40,0.40,0.40}{##1}}}
\expandafter\def\csname PY@tok@mf\endcsname{\def\PY@tc##1{\textcolor[rgb]{0.40,0.40,0.40}{##1}}}
\expandafter\def\csname PY@tok@mh\endcsname{\def\PY@tc##1{\textcolor[rgb]{0.40,0.40,0.40}{##1}}}
\expandafter\def\csname PY@tok@mi\endcsname{\def\PY@tc##1{\textcolor[rgb]{0.40,0.40,0.40}{##1}}}
\expandafter\def\csname PY@tok@il\endcsname{\def\PY@tc##1{\textcolor[rgb]{0.40,0.40,0.40}{##1}}}
\expandafter\def\csname PY@tok@mo\endcsname{\def\PY@tc##1{\textcolor[rgb]{0.40,0.40,0.40}{##1}}}
\expandafter\def\csname PY@tok@ch\endcsname{\let\PY@it=\textit\def\PY@tc##1{\textcolor[rgb]{0.25,0.50,0.50}{##1}}}
\expandafter\def\csname PY@tok@cm\endcsname{\let\PY@it=\textit\def\PY@tc##1{\textcolor[rgb]{0.25,0.50,0.50}{##1}}}
\expandafter\def\csname PY@tok@cpf\endcsname{\let\PY@it=\textit\def\PY@tc##1{\textcolor[rgb]{0.25,0.50,0.50}{##1}}}
\expandafter\def\csname PY@tok@c1\endcsname{\let\PY@it=\textit\def\PY@tc##1{\textcolor[rgb]{0.25,0.50,0.50}{##1}}}
\expandafter\def\csname PY@tok@cs\endcsname{\let\PY@it=\textit\def\PY@tc##1{\textcolor[rgb]{0.25,0.50,0.50}{##1}}}

\def\PYZbs{\char`\\}
\def\PYZus{\char`\_}
\def\PYZob{\char`\{}
\def\PYZcb{\char`\}}
\def\PYZca{\char`\^}
\def\PYZam{\char`\&}
\def\PYZlt{\char`\<}
\def\PYZgt{\char`\>}
\def\PYZsh{\char`\#}
\def\PYZpc{\char`\%}
\def\PYZdl{\char`\$}
\def\PYZhy{\char`\-}
\def\PYZsq{\char`\'}
\def\PYZdq{\char`\"}
\def\PYZti{\char`\~}
% for compatibility with earlier versions
\def\PYZat{@}
\def\PYZlb{[}
\def\PYZrb{]}
\makeatother


    % Exact colors from NB
    \definecolor{incolor}{rgb}{0.0, 0.0, 0.5}
    \definecolor{outcolor}{rgb}{0.545, 0.0, 0.0}



    
    % Prevent overflowing lines due to hard-to-break entities
    \sloppy 
    % Setup hyperref package
    \hypersetup{
      breaklinks=true,  % so long urls are correctly broken across lines
      colorlinks=true,
      urlcolor=urlcolor,
      linkcolor=linkcolor,
      citecolor=citecolor,
      }
    % Slightly bigger margins than the latex defaults
    
    \geometry{verbose,tmargin=1in,bmargin=1in,lmargin=1in,rmargin=1in}
    
    

    \begin{document}
    
    
    \maketitle
    
    

    
    \hypertarget{programming-assignment}{%
\section{Programming Assignment}\label{programming-assignment}}

\hypertarget{saving-and-loading-models-with-application-to-the-eurosat-dataset}{%
\subsection{Saving and loading models, with application to the EuroSat
dataset}\label{saving-and-loading-models-with-application-to-the-eurosat-dataset}}

\hypertarget{instructions}{%
\subsubsection{Instructions}\label{instructions}}

In this notebook, you will create a neural network that classifies land
uses and land covers from satellite imagery. You will save your model
using Tensorflow's callbacks and reload it later. You will also load in
a pre-trained neural network classifier and compare performance with it.

Some code cells are provided for you in the notebook. You should avoid
editing provided code, and make sure to execute the cells in order to
avoid unexpected errors. Some cells begin with the line:

\texttt{\#\#\#\#\ GRADED\ CELL\ \#\#\#\#}

Don't move or edit this first line - this is what the automatic grader
looks for to recognise graded cells. These cells require you to write
your own code to complete them, and are automatically graded when you
submit the notebook. Don't edit the function name or signature provided
in these cells, otherwise the automatic grader might not function
properly. Inside these graded cells, you can use any functions or
classes that are imported below, but make sure you don't use any
variables that are outside the scope of the function.

\hypertarget{how-to-submit}{%
\subsubsection{How to submit}\label{how-to-submit}}

Complete all the tasks you are asked for in the worksheet. When you have
finished and are happy with your code, press the \textbf{Submit
Assignment} button at the top of this notebook.

\hypertarget{lets-get-started}{%
\subsubsection{Let's get started!}\label{lets-get-started}}

We'll start running some imports, and loading the dataset. Do not edit
the existing imports in the following cell. If you would like to make
further Tensorflow imports, you should add them here.

    \begin{Verbatim}[commandchars=\\\{\}]
{\color{incolor}In [{\color{incolor}1}]:} \PY{c+c1}{\PYZsh{}\PYZsh{}\PYZsh{}\PYZsh{} PACKAGE IMPORTS \PYZsh{}\PYZsh{}\PYZsh{}\PYZsh{}}
        
        \PY{c+c1}{\PYZsh{} Run this cell first to import all required packages. Do not make any imports elsewhere in the notebook}
        
        \PY{k+kn}{import} \PY{n+nn}{tensorflow} \PY{k}{as} \PY{n+nn}{tf}
        \PY{k+kn}{from} \PY{n+nn}{tensorflow}\PY{n+nn}{.}\PY{n+nn}{keras}\PY{n+nn}{.}\PY{n+nn}{preprocessing}\PY{n+nn}{.}\PY{n+nn}{image} \PY{k}{import} \PY{n}{load\PYZus{}img}\PY{p}{,} \PY{n}{img\PYZus{}to\PYZus{}array}
        \PY{k+kn}{from} \PY{n+nn}{tensorflow}\PY{n+nn}{.}\PY{n+nn}{keras}\PY{n+nn}{.}\PY{n+nn}{models} \PY{k}{import} \PY{n}{Sequential}\PY{p}{,} \PY{n}{load\PYZus{}model}
        \PY{k+kn}{from} \PY{n+nn}{tensorflow}\PY{n+nn}{.}\PY{n+nn}{keras}\PY{n+nn}{.}\PY{n+nn}{layers} \PY{k}{import} \PY{n}{Dense}\PY{p}{,} \PY{n}{Flatten}\PY{p}{,} \PY{n}{Conv2D}\PY{p}{,} \PY{n}{MaxPooling2D}
        \PY{k+kn}{from} \PY{n+nn}{tensorflow}\PY{n+nn}{.}\PY{n+nn}{keras}\PY{n+nn}{.}\PY{n+nn}{callbacks} \PY{k}{import} \PY{n}{ModelCheckpoint}\PY{p}{,} \PY{n}{EarlyStopping}
        \PY{k+kn}{import} \PY{n+nn}{os}
        \PY{k+kn}{import} \PY{n+nn}{numpy} \PY{k}{as} \PY{n+nn}{np}
        \PY{k+kn}{import} \PY{n+nn}{pandas} \PY{k}{as} \PY{n+nn}{pd}
        
        \PY{c+c1}{\PYZsh{} If you would like to make further imports from tensorflow, add them here}
\end{Verbatim}

    \begin{figure}
\centering
\includegraphics{data/eurosat_overview.jpg}
\caption{EuroSAT overview image}
\end{figure}

\hypertarget{the-eurosat-dataset}{%
\paragraph{The EuroSAT dataset}\label{the-eurosat-dataset}}

In this assignment, you will use the
\href{https://github.com/phelber/EuroSAT}{EuroSAT dataset}. It consists
of 27000 labelled Sentinel-2 satellite images of different land uses:
residential, industrial, highway, river, forest, pasture, herbaceous
vegetation, annual crop, permanent crop and sea/lake. For a reference,
see the following papers: - Eurosat: A novel dataset and deep learning
benchmark for land use and land cover classification. Patrick Helber,
Benjamin Bischke, Andreas Dengel, Damian Borth. IEEE Journal of Selected
Topics in Applied Earth Observations and Remote Sensing, 2019. -
Introducing EuroSAT: A Novel Dataset and Deep Learning Benchmark for
Land Use and Land Cover Classification. Patrick Helber, Benjamin
Bischke, Andreas Dengel. 2018 IEEE International Geoscience and Remote
Sensing Symposium, 2018.

Your goal is to construct a neural network that classifies a satellite
image into one of these 10 classes, as well as applying some of the
saving and loading techniques you have learned in the previous sessions.

    \hypertarget{import-the-data}{%
\paragraph{Import the data}\label{import-the-data}}

The dataset you will train your model on is a subset of the total data,
with 4000 training images and 1000 testing images, with roughly equal
numbers of each class. The code to import the data is provided below.

    \begin{Verbatim}[commandchars=\\\{\}]
{\color{incolor}In [{\color{incolor}2}]:} \PY{c+c1}{\PYZsh{} Run this cell to import the Eurosat data}
        
        \PY{k}{def} \PY{n+nf}{load\PYZus{}eurosat\PYZus{}data}\PY{p}{(}\PY{p}{)}\PY{p}{:}
            \PY{n}{data\PYZus{}dir} \PY{o}{=} \PY{l+s+s1}{\PYZsq{}}\PY{l+s+s1}{data/}\PY{l+s+s1}{\PYZsq{}}
            \PY{n}{x\PYZus{}train} \PY{o}{=} \PY{n}{np}\PY{o}{.}\PY{n}{load}\PY{p}{(}\PY{n}{os}\PY{o}{.}\PY{n}{path}\PY{o}{.}\PY{n}{join}\PY{p}{(}\PY{n}{data\PYZus{}dir}\PY{p}{,} \PY{l+s+s1}{\PYZsq{}}\PY{l+s+s1}{x\PYZus{}train.npy}\PY{l+s+s1}{\PYZsq{}}\PY{p}{)}\PY{p}{)}
            \PY{n}{y\PYZus{}train} \PY{o}{=} \PY{n}{np}\PY{o}{.}\PY{n}{load}\PY{p}{(}\PY{n}{os}\PY{o}{.}\PY{n}{path}\PY{o}{.}\PY{n}{join}\PY{p}{(}\PY{n}{data\PYZus{}dir}\PY{p}{,} \PY{l+s+s1}{\PYZsq{}}\PY{l+s+s1}{y\PYZus{}train.npy}\PY{l+s+s1}{\PYZsq{}}\PY{p}{)}\PY{p}{)}
            \PY{n}{x\PYZus{}test}  \PY{o}{=} \PY{n}{np}\PY{o}{.}\PY{n}{load}\PY{p}{(}\PY{n}{os}\PY{o}{.}\PY{n}{path}\PY{o}{.}\PY{n}{join}\PY{p}{(}\PY{n}{data\PYZus{}dir}\PY{p}{,} \PY{l+s+s1}{\PYZsq{}}\PY{l+s+s1}{x\PYZus{}test.npy}\PY{l+s+s1}{\PYZsq{}}\PY{p}{)}\PY{p}{)}
            \PY{n}{y\PYZus{}test}  \PY{o}{=} \PY{n}{np}\PY{o}{.}\PY{n}{load}\PY{p}{(}\PY{n}{os}\PY{o}{.}\PY{n}{path}\PY{o}{.}\PY{n}{join}\PY{p}{(}\PY{n}{data\PYZus{}dir}\PY{p}{,} \PY{l+s+s1}{\PYZsq{}}\PY{l+s+s1}{y\PYZus{}test.npy}\PY{l+s+s1}{\PYZsq{}}\PY{p}{)}\PY{p}{)}
            \PY{k}{return} \PY{p}{(}\PY{n}{x\PYZus{}train}\PY{p}{,} \PY{n}{y\PYZus{}train}\PY{p}{)}\PY{p}{,} \PY{p}{(}\PY{n}{x\PYZus{}test}\PY{p}{,} \PY{n}{y\PYZus{}test}\PY{p}{)}
        
        \PY{p}{(}\PY{n}{x\PYZus{}train}\PY{p}{,} \PY{n}{y\PYZus{}train}\PY{p}{)}\PY{p}{,} \PY{p}{(}\PY{n}{x\PYZus{}test}\PY{p}{,} \PY{n}{y\PYZus{}test}\PY{p}{)} \PY{o}{=} \PY{n}{load\PYZus{}eurosat\PYZus{}data}\PY{p}{(}\PY{p}{)}
        \PY{n}{x\PYZus{}train} \PY{o}{=} \PY{n}{x\PYZus{}train} \PY{o}{/} \PY{l+m+mf}{255.0}
        \PY{n}{x\PYZus{}test} \PY{o}{=} \PY{n}{x\PYZus{}test} \PY{o}{/} \PY{l+m+mf}{255.0}
\end{Verbatim}

    \hypertarget{build-the-neural-network-model}{%
\paragraph{Build the neural network
model}\label{build-the-neural-network-model}}

    You can now construct a model to fit to the data. Using the Sequential
API, build your model according to the following specifications:

\begin{itemize}
\tightlist
\item
  The model should use the input\_shape in the function argument to set
  the input size in the first layer.
\item
  The first layer should be a Conv2D layer with 16 filters, a 3x3 kernel
  size, a ReLU activation function and `SAME' padding. Name this layer
  `conv\_1'.
\item
  The second layer should also be a Conv2D layer with 8 filters, a 3x3
  kernel size, a ReLU activation function and `SAME' padding. Name this
  layer `conv\_2'.
\item
  The third layer should be a MaxPooling2D layer with a pooling window
  size of 8x8. Name this layer `pool\_1'.
\item
  The fourth layer should be a Flatten layer, named `flatten'.
\item
  The fifth layer should be a Dense layer with 32 units, a ReLU
  activation. Name this layer `dense\_1'.
\item
  The sixth and final layer should be a Dense layer with 10 units and
  softmax activation. Name this layer `dense\_2'.
\end{itemize}

In total, the network should have 6 layers.

    \begin{Verbatim}[commandchars=\\\{\}]
{\color{incolor}In [{\color{incolor}15}]:} \PY{c+c1}{\PYZsh{}\PYZsh{}\PYZsh{}\PYZsh{} GRADED CELL \PYZsh{}\PYZsh{}\PYZsh{}\PYZsh{}}
         
         \PY{c+c1}{\PYZsh{} Complete the following function. }
         \PY{c+c1}{\PYZsh{} Make sure to not change the function name or arguments.}
         
         \PY{k}{def} \PY{n+nf}{get\PYZus{}new\PYZus{}model}\PY{p}{(}\PY{n}{input\PYZus{}shape}\PY{p}{)}\PY{p}{:}
             \PY{l+s+sd}{\PYZdq{}\PYZdq{}\PYZdq{}}
         \PY{l+s+sd}{    This function should build a Sequential model according to the above specification. Ensure the }
         \PY{l+s+sd}{    weights are initialised by providing the input\PYZus{}shape argument in the first layer, given by the}
         \PY{l+s+sd}{    function argument.}
         \PY{l+s+sd}{    Your function should also compile the model with the Adam optimiser, sparse categorical cross}
         \PY{l+s+sd}{    entropy loss function, and a single accuracy metric.}
         \PY{l+s+sd}{    \PYZdq{}\PYZdq{}\PYZdq{}}
             \PY{n}{model}\PY{o}{=}\PY{n}{Sequential}\PY{p}{(}\PY{p}{[}
                 \PY{n}{Conv2D}\PY{p}{(}\PY{l+m+mi}{16}\PY{p}{,}\PY{p}{(}\PY{l+m+mi}{3}\PY{p}{,}\PY{l+m+mi}{3}\PY{p}{)}\PY{p}{,}\PY{n}{activation}\PY{o}{=}\PY{l+s+s1}{\PYZsq{}}\PY{l+s+s1}{relu}\PY{l+s+s1}{\PYZsq{}}\PY{p}{,}\PY{n}{padding}\PY{o}{=}\PY{l+s+s1}{\PYZsq{}}\PY{l+s+s1}{SAME}\PY{l+s+s1}{\PYZsq{}}\PY{p}{,}\PY{n}{name}\PY{o}{=}\PY{l+s+s1}{\PYZsq{}}\PY{l+s+s1}{conv\PYZus{}1}\PY{l+s+s1}{\PYZsq{}}\PY{p}{,}\PY{n}{input\PYZus{}shape}\PY{o}{=}\PY{n}{input\PYZus{}shape}\PY{p}{)}\PY{p}{,}
                 \PY{n}{Conv2D}\PY{p}{(}\PY{l+m+mi}{8}\PY{p}{,}\PY{p}{(}\PY{l+m+mi}{3}\PY{p}{,}\PY{l+m+mi}{3}\PY{p}{)}\PY{p}{,}\PY{n}{activation}\PY{o}{=}\PY{l+s+s1}{\PYZsq{}}\PY{l+s+s1}{relu}\PY{l+s+s1}{\PYZsq{}}\PY{p}{,}\PY{n}{padding}\PY{o}{=}\PY{l+s+s1}{\PYZsq{}}\PY{l+s+s1}{SAME}\PY{l+s+s1}{\PYZsq{}}\PY{p}{,}\PY{n}{name}\PY{o}{=}\PY{l+s+s1}{\PYZsq{}}\PY{l+s+s1}{conv\PYZus{}2}\PY{l+s+s1}{\PYZsq{}}\PY{p}{)}\PY{p}{,}
                 \PY{n}{MaxPooling2D}\PY{p}{(}\PY{p}{(}\PY{l+m+mi}{8}\PY{p}{,}\PY{l+m+mi}{8}\PY{p}{)}\PY{p}{,}\PY{n}{name}\PY{o}{=}\PY{l+s+s1}{\PYZsq{}}\PY{l+s+s1}{pool\PYZus{}1}\PY{l+s+s1}{\PYZsq{}}\PY{p}{)}\PY{p}{,}
                 \PY{n}{Flatten}\PY{p}{(}\PY{n}{name}\PY{o}{=}\PY{l+s+s1}{\PYZsq{}}\PY{l+s+s1}{flatten}\PY{l+s+s1}{\PYZsq{}}\PY{p}{)}\PY{p}{,}
                 \PY{n}{Dense}\PY{p}{(}\PY{l+m+mi}{32}\PY{p}{,}\PY{n}{activation}\PY{o}{=}\PY{l+s+s1}{\PYZsq{}}\PY{l+s+s1}{relu}\PY{l+s+s1}{\PYZsq{}}\PY{p}{,}\PY{n}{name}\PY{o}{=}\PY{l+s+s1}{\PYZsq{}}\PY{l+s+s1}{dense\PYZus{}1}\PY{l+s+s1}{\PYZsq{}}\PY{p}{)}\PY{p}{,}
                 \PY{n}{Dense}\PY{p}{(}\PY{l+m+mi}{10}\PY{p}{,}\PY{n}{activation}\PY{o}{=}\PY{l+s+s1}{\PYZsq{}}\PY{l+s+s1}{softmax}\PY{l+s+s1}{\PYZsq{}}\PY{p}{,}\PY{n}{name}\PY{o}{=}\PY{l+s+s1}{\PYZsq{}}\PY{l+s+s1}{dense\PYZus{}2}\PY{l+s+s1}{\PYZsq{}}\PY{p}{)}
                 \PY{p}{]}\PY{p}{)}
             \PY{k}{return} \PY{n}{model}
         
             
\end{Verbatim}

    \hypertarget{compile-and-evaluate-the-model}{%
\paragraph{Compile and evaluate the
model}\label{compile-and-evaluate-the-model}}

    \begin{Verbatim}[commandchars=\\\{\}]
{\color{incolor}In [{\color{incolor}16}]:} \PY{c+c1}{\PYZsh{} Run your function to create the model}
         
         \PY{n}{model} \PY{o}{=} \PY{n}{get\PYZus{}new\PYZus{}model}\PY{p}{(}\PY{n}{x\PYZus{}train}\PY{p}{[}\PY{l+m+mi}{0}\PY{p}{]}\PY{o}{.}\PY{n}{shape}\PY{p}{)}
\end{Verbatim}

    \begin{Verbatim}[commandchars=\\\{\}]
{\color{incolor}In [{\color{incolor}17}]:} \PY{c+c1}{\PYZsh{} Run this cell to define a function to evaluate a model\PYZsq{}s test accuracy}
         
         \PY{k}{def} \PY{n+nf}{get\PYZus{}test\PYZus{}accuracy}\PY{p}{(}\PY{n}{model}\PY{p}{,} \PY{n}{x\PYZus{}test}\PY{p}{,} \PY{n}{y\PYZus{}test}\PY{p}{)}\PY{p}{:}
             \PY{l+s+sd}{\PYZdq{}\PYZdq{}\PYZdq{}Test model classification accuracy\PYZdq{}\PYZdq{}\PYZdq{}}
             \PY{n}{test\PYZus{}loss}\PY{p}{,} \PY{n}{test\PYZus{}acc} \PY{o}{=} \PY{n}{model}\PY{o}{.}\PY{n}{evaluate}\PY{p}{(}\PY{n}{x}\PY{o}{=}\PY{n}{x\PYZus{}test}\PY{p}{,} \PY{n}{y}\PY{o}{=}\PY{n}{y\PYZus{}test}\PY{p}{,} \PY{n}{verbose}\PY{o}{=}\PY{l+m+mi}{0}\PY{p}{)}
             \PY{n+nb}{print}\PY{p}{(}\PY{l+s+s1}{\PYZsq{}}\PY{l+s+s1}{accuracy: }\PY{l+s+si}{\PYZob{}acc:0.3f\PYZcb{}}\PY{l+s+s1}{\PYZsq{}}\PY{o}{.}\PY{n}{format}\PY{p}{(}\PY{n}{acc}\PY{o}{=}\PY{n}{test\PYZus{}acc}\PY{p}{)}\PY{p}{)}
\end{Verbatim}

    \begin{Verbatim}[commandchars=\\\{\}]
{\color{incolor}In [{\color{incolor}18}]:} \PY{c+c1}{\PYZsh{} Print the model summary and calculate its initialised test accuracy}
         
         \PY{n}{model}\PY{o}{.}\PY{n}{summary}\PY{p}{(}\PY{p}{)}
\end{Verbatim}

    \begin{Verbatim}[commandchars=\\\{\}]
Model: "sequential\_4"
\_\_\_\_\_\_\_\_\_\_\_\_\_\_\_\_\_\_\_\_\_\_\_\_\_\_\_\_\_\_\_\_\_\_\_\_\_\_\_\_\_\_\_\_\_\_\_\_\_\_\_\_\_\_\_\_\_\_\_\_\_\_\_\_\_
Layer (type)                 Output Shape              Param \#   
=================================================================
conv\_1 (Conv2D)              (None, 64, 64, 16)        448       
\_\_\_\_\_\_\_\_\_\_\_\_\_\_\_\_\_\_\_\_\_\_\_\_\_\_\_\_\_\_\_\_\_\_\_\_\_\_\_\_\_\_\_\_\_\_\_\_\_\_\_\_\_\_\_\_\_\_\_\_\_\_\_\_\_
conv\_2 (Conv2D)              (None, 64, 64, 8)         1160      
\_\_\_\_\_\_\_\_\_\_\_\_\_\_\_\_\_\_\_\_\_\_\_\_\_\_\_\_\_\_\_\_\_\_\_\_\_\_\_\_\_\_\_\_\_\_\_\_\_\_\_\_\_\_\_\_\_\_\_\_\_\_\_\_\_
pool\_1 (MaxPooling2D)        (None, 8, 8, 8)           0         
\_\_\_\_\_\_\_\_\_\_\_\_\_\_\_\_\_\_\_\_\_\_\_\_\_\_\_\_\_\_\_\_\_\_\_\_\_\_\_\_\_\_\_\_\_\_\_\_\_\_\_\_\_\_\_\_\_\_\_\_\_\_\_\_\_
flatten (Flatten)            (None, 512)               0         
\_\_\_\_\_\_\_\_\_\_\_\_\_\_\_\_\_\_\_\_\_\_\_\_\_\_\_\_\_\_\_\_\_\_\_\_\_\_\_\_\_\_\_\_\_\_\_\_\_\_\_\_\_\_\_\_\_\_\_\_\_\_\_\_\_
dense\_1 (Dense)              (None, 32)                16416     
\_\_\_\_\_\_\_\_\_\_\_\_\_\_\_\_\_\_\_\_\_\_\_\_\_\_\_\_\_\_\_\_\_\_\_\_\_\_\_\_\_\_\_\_\_\_\_\_\_\_\_\_\_\_\_\_\_\_\_\_\_\_\_\_\_
dense\_2 (Dense)              (None, 10)                330       
=================================================================
Total params: 18,354
Trainable params: 18,354
Non-trainable params: 0
\_\_\_\_\_\_\_\_\_\_\_\_\_\_\_\_\_\_\_\_\_\_\_\_\_\_\_\_\_\_\_\_\_\_\_\_\_\_\_\_\_\_\_\_\_\_\_\_\_\_\_\_\_\_\_\_\_\_\_\_\_\_\_\_\_

    \end{Verbatim}

    \hypertarget{create-checkpoints-to-save-model-during-training-with-a-criterion}{%
\paragraph{Create checkpoints to save model during training, with a
criterion}\label{create-checkpoints-to-save-model-during-training-with-a-criterion}}

You will now create three callbacks: -
\texttt{checkpoint\_every\_epoch}: checkpoint that saves the model
weights every epoch during training - \texttt{checkpoint\_best\_only}:
checkpoint that saves only the weights with the highest validation
accuracy. Use the testing data as the validation data. -
\texttt{early\_stopping}: early stopping object that ends training if
the validation accuracy has not improved in 3 epochs.

    \begin{Verbatim}[commandchars=\\\{\}]
{\color{incolor}In [{\color{incolor}50}]:} \PY{c+c1}{\PYZsh{}\PYZsh{}\PYZsh{}\PYZsh{} GRADED CELL \PYZsh{}\PYZsh{}\PYZsh{}\PYZsh{}}
         
         \PY{c+c1}{\PYZsh{} Complete the following functions. }
         \PY{c+c1}{\PYZsh{} Make sure to not change the function names or arguments.}
         
         \PY{k}{def} \PY{n+nf}{get\PYZus{}checkpoint\PYZus{}every\PYZus{}epoch}\PY{p}{(}\PY{p}{)}\PY{p}{:}
             \PY{l+s+sd}{\PYZdq{}\PYZdq{}\PYZdq{}}
         \PY{l+s+sd}{    This function should return a ModelCheckpoint object that:}
         \PY{l+s+sd}{    \PYZhy{} saves the weights only at the end of every epoch}
         \PY{l+s+sd}{    \PYZhy{} saves into a directory called \PYZsq{}checkpoints\PYZus{}every\PYZus{}epoch\PYZsq{} inside the current working directory}
         \PY{l+s+sd}{    \PYZhy{} generates filenames in that directory like \PYZsq{}checkpoint\PYZus{}XXX\PYZsq{} where}
         \PY{l+s+sd}{      XXX is the epoch number formatted to have three digits, e.g. 001, 002, 003, etc.}
         \PY{l+s+sd}{      }
         \PY{l+s+sd}{    \PYZdq{}\PYZdq{}\PYZdq{}}
             \PY{n}{checkpoint\PYZus{}path}\PY{o}{=}\PY{l+s+s1}{\PYZsq{}}\PY{l+s+s1}{checkpoints\PYZus{}every\PYZus{}epoch/checkpoint\PYZus{}}\PY{l+s+si}{\PYZob{}epoch:03d\PYZcb{}}\PY{l+s+s1}{\PYZsq{}}
             \PY{k}{return} \PY{n}{ModelCheckpoint}\PY{p}{(}\PY{n}{filepath}\PY{o}{=}\PY{n}{checkpoint\PYZus{}path}\PY{p}{,}\PY{n}{save\PYZus{}weights\PYZus{}only}\PY{o}{=}\PY{k+kc}{True}\PY{p}{,}\PY{n}{verbose}\PY{o}{=}\PY{l+m+mi}{1}\PY{p}{)}
             
         
         
         \PY{k}{def} \PY{n+nf}{get\PYZus{}checkpoint\PYZus{}best\PYZus{}only}\PY{p}{(}\PY{p}{)}\PY{p}{:}
             \PY{l+s+sd}{\PYZdq{}\PYZdq{}\PYZdq{}}
         \PY{l+s+sd}{    This function should return a ModelCheckpoint object that:}
         \PY{l+s+sd}{    \PYZhy{} saves only the weights that generate the highest validation (testing) accuracy}
         \PY{l+s+sd}{    \PYZhy{} saves into a directory called \PYZsq{}checkpoints\PYZus{}best\PYZus{}only\PYZsq{} inside the current working directory}
         \PY{l+s+sd}{    \PYZhy{} generates a file called \PYZsq{}checkpoints\PYZus{}best\PYZus{}only/checkpoint\PYZsq{} }
         \PY{l+s+sd}{    \PYZdq{}\PYZdq{}\PYZdq{}}
             \PY{n}{checkpoint\PYZus{}path}\PY{o}{=}\PY{l+s+s1}{\PYZsq{}}\PY{l+s+s1}{checkpoints\PYZus{}best\PYZus{}only/checkpoints\PYZus{}best\PYZus{}only/checkpoint}\PY{l+s+s1}{\PYZsq{}}
             \PY{k}{return} \PY{n}{ModelCheckpoint}\PY{p}{(}\PY{n}{checkpoint\PYZus{}path}\PY{p}{,}\PY{n}{save\PYZus{}weights\PYZus{}only}\PY{o}{=}\PY{k+kc}{True}\PY{p}{,}\PY{n}{mointor}\PY{o}{=}\PY{l+s+s1}{\PYZsq{}}\PY{l+s+s1}{val\PYZus{}accuracy}\PY{l+s+s1}{\PYZsq{}}\PY{p}{,}\PY{n}{mode}\PY{o}{=}\PY{l+s+s1}{\PYZsq{}}\PY{l+s+s1}{max}\PY{l+s+s1}{\PYZsq{}}\PY{p}{,}\PY{n}{save\PYZus{}best\PYZus{}only}\PY{o}{=}\PY{k+kc}{True}\PY{p}{,}\PY{n}{verbose}\PY{o}{=}\PY{l+m+mi}{1}\PY{p}{)}
         
             
             
\end{Verbatim}

    \begin{Verbatim}[commandchars=\\\{\}]
{\color{incolor}In [{\color{incolor}51}]:} \PY{c+c1}{\PYZsh{}\PYZsh{}\PYZsh{}\PYZsh{} GRADED CELL \PYZsh{}\PYZsh{}\PYZsh{}\PYZsh{}}
         
         \PY{c+c1}{\PYZsh{} Complete the following function. }
         \PY{c+c1}{\PYZsh{} Make sure to not change the function name or arguments.}
         
         \PY{k}{def} \PY{n+nf}{get\PYZus{}early\PYZus{}stopping}\PY{p}{(}\PY{p}{)}\PY{p}{:}
             \PY{l+s+sd}{\PYZdq{}\PYZdq{}\PYZdq{}}
         \PY{l+s+sd}{    This function should return an EarlyStopping callback that stops training when}
         \PY{l+s+sd}{    the validation (testing) accuracy has not improved in the last 3 epochs.}
         \PY{l+s+sd}{    HINT: use the EarlyStopping callback with the correct \PYZsq{}monitor\PYZsq{} and \PYZsq{}patience\PYZsq{}}
         \PY{l+s+sd}{    \PYZdq{}\PYZdq{}\PYZdq{}}
             \PY{k}{return} \PY{n}{EarlyStopping}\PY{p}{(}\PY{n}{monitor}\PY{o}{=}\PY{l+s+s1}{\PYZsq{}}\PY{l+s+s1}{val\PYZus{}accuracy}\PY{l+s+s1}{\PYZsq{}}\PY{p}{,}\PY{n}{patience}\PY{o}{=}\PY{l+m+mi}{3}\PY{p}{)}
             
\end{Verbatim}

    \begin{Verbatim}[commandchars=\\\{\}]
{\color{incolor}In [{\color{incolor}52}]:} \PY{c+c1}{\PYZsh{} Run this cell to create the callbacks}
         
         \PY{n}{checkpoint\PYZus{}every\PYZus{}epoch} \PY{o}{=} \PY{n}{get\PYZus{}checkpoint\PYZus{}every\PYZus{}epoch}\PY{p}{(}\PY{p}{)}
         \PY{n}{checkpoint\PYZus{}best\PYZus{}only} \PY{o}{=} \PY{n}{get\PYZus{}checkpoint\PYZus{}best\PYZus{}only}\PY{p}{(}\PY{p}{)}
         \PY{n}{early\PYZus{}stopping} \PY{o}{=} \PY{n}{get\PYZus{}early\PYZus{}stopping}\PY{p}{(}\PY{p}{)}
\end{Verbatim}

    \begin{Verbatim}[commandchars=\\\{\}]
{\color{incolor}In [{\color{incolor}53}]:} \PY{n}{checkpoint\PYZus{}best\PYZus{}only}
\end{Verbatim}

\begin{Verbatim}[commandchars=\\\{\}]
{\color{outcolor}Out[{\color{outcolor}53}]:} <tensorflow.python.keras.callbacks.ModelCheckpoint at 0x7f28886a4cf8>
\end{Verbatim}
            
    \hypertarget{train-model-using-the-callbacks}{%
\paragraph{Train model using the
callbacks}\label{train-model-using-the-callbacks}}

Now, you will train the model using the three callbacks you created. If
you created the callbacks correctly, three things should happen: - At
the end of every epoch, the model weights are saved into a directory
called \texttt{checkpoints\_every\_epoch} - At the end of every epoch,
the model weights are saved into a directory called
\texttt{checkpoints\_best\_only} \textbf{only} if those weights lead to
the highest test accuracy - Training stops when the testing accuracy has
not improved in three epochs.

You should then have two directories: - A directory called
\texttt{checkpoints\_every\_epoch} containing filenames that include
\texttt{checkpoint\_001}, \texttt{checkpoint\_002}, etc with the
\texttt{001}, \texttt{002} corresponding to the epoch - A directory
called \texttt{checkpoints\_best\_only} containing filenames that
include \texttt{checkpoint}, which contain only the weights leading to
the highest testing accuracy

    \begin{Verbatim}[commandchars=\\\{\}]
{\color{incolor}In [{\color{incolor}54}]:} \PY{c+c1}{\PYZsh{} Train model using the callbacks you just created}
         \PY{n}{model}\PY{o}{.}\PY{n}{compile}\PY{p}{(}\PY{n}{optimizer}\PY{o}{=}\PY{l+s+s1}{\PYZsq{}}\PY{l+s+s1}{adam}\PY{l+s+s1}{\PYZsq{}}\PY{p}{,}\PY{n}{loss}\PY{o}{=}\PY{l+s+s1}{\PYZsq{}}\PY{l+s+s1}{sparse\PYZus{}categorical\PYZus{}crossentropy}\PY{l+s+s1}{\PYZsq{}}\PY{p}{,}\PY{n}{metrics}\PY{o}{=}\PY{p}{[}\PY{l+s+s1}{\PYZsq{}}\PY{l+s+s1}{accuracy}\PY{l+s+s1}{\PYZsq{}}\PY{p}{]}\PY{p}{)}
         \PY{n}{callbacks} \PY{o}{=} \PY{p}{[}\PY{n}{checkpoint\PYZus{}every\PYZus{}epoch}\PY{p}{,} \PY{n}{checkpoint\PYZus{}best\PYZus{}only}\PY{p}{,} \PY{n}{early\PYZus{}stopping}\PY{p}{]}
         \PY{n}{model}\PY{o}{.}\PY{n}{fit}\PY{p}{(}\PY{n}{x\PYZus{}train}\PY{p}{,} \PY{n}{y\PYZus{}train}\PY{p}{,} \PY{n}{epochs}\PY{o}{=}\PY{l+m+mi}{3}\PY{p}{,} \PY{n}{validation\PYZus{}data}\PY{o}{=}\PY{p}{(}\PY{n}{x\PYZus{}test}\PY{p}{,} \PY{n}{y\PYZus{}test}\PY{p}{)}\PY{p}{,} \PY{n}{callbacks}\PY{o}{=}\PY{n}{callbacks}\PY{p}{)}
\end{Verbatim}

    \begin{Verbatim}[commandchars=\\\{\}]
Train on 4000 samples, validate on 1000 samples
Epoch 1/3
3968/4000 [============================>.] - ETA: 0s - loss: 0.5835 - accuracy: 0.7873
Epoch 00001: saving model to checkpoints\_every\_epoch/checkpoint\_001

Epoch 00001: val\_loss improved from -inf to 0.76425, saving model to checkpoints\_best\_only/checkpoints\_best\_only/checkpoint
4000/4000 [==============================] - 84s 21ms/sample - loss: 0.5826 - accuracy: 0.7875 - val\_loss: 0.7642 - val\_accuracy: 0.7220
Epoch 2/3
3968/4000 [============================>.] - ETA: 0s - loss: 0.5984 - accuracy: 0.7883
Epoch 00002: saving model to checkpoints\_every\_epoch/checkpoint\_002

Epoch 00002: val\_loss did not improve from 0.76425
4000/4000 [==============================] - 79s 20ms/sample - loss: 0.6007 - accuracy: 0.7878 - val\_loss: 0.7417 - val\_accuracy: 0.7390
Epoch 3/3
3968/4000 [============================>.] - ETA: 0s - loss: 0.5645 - accuracy: 0.8009
Epoch 00003: saving model to checkpoints\_every\_epoch/checkpoint\_003

Epoch 00003: val\_loss did not improve from 0.76425
4000/4000 [==============================] - 81s 20ms/sample - loss: 0.5650 - accuracy: 0.8000 - val\_loss: 0.7543 - val\_accuracy: 0.7440

    \end{Verbatim}

\begin{Verbatim}[commandchars=\\\{\}]
{\color{outcolor}Out[{\color{outcolor}54}]:} <tensorflow.python.keras.callbacks.History at 0x7f28885d86a0>
\end{Verbatim}
            
    \hypertarget{create-new-instance-of-model-and-load-on-both-sets-of-weights}{%
\paragraph{Create new instance of model and load on both sets of
weights}\label{create-new-instance-of-model-and-load-on-both-sets-of-weights}}

Now you will use the weights you just saved in a fresh model. You should
create two functions, both of which take a freshly instantiated model
instance: - \texttt{model\_last\_epoch} should contain the weights from
the latest saved epoch - \texttt{model\_best\_epoch} should contain the
weights from the saved epoch with the highest testing accuracy

\emph{Hint: use the} \texttt{tf.train.latest\_checkpoint} \emph{function
to get the filename of the latest saved checkpoint file. Check the docs}
\href{https://www.tensorflow.org/api_docs/python/tf/train/latest_checkpoint}{\emph{here}}.

    \begin{Verbatim}[commandchars=\\\{\}]
{\color{incolor}In [{\color{incolor}55}]:} \PY{c+c1}{\PYZsh{}\PYZsh{}\PYZsh{}\PYZsh{} GRADED CELL \PYZsh{}\PYZsh{}\PYZsh{}\PYZsh{}}
         
         \PY{c+c1}{\PYZsh{} Complete the following functions. }
         \PY{c+c1}{\PYZsh{} Make sure to not change the function name or arguments.}
         
         \PY{k}{def} \PY{n+nf}{get\PYZus{}model\PYZus{}last\PYZus{}epoch}\PY{p}{(}\PY{n}{model}\PY{p}{)}\PY{p}{:}
             \PY{l+s+sd}{\PYZdq{}\PYZdq{}\PYZdq{}}
         \PY{l+s+sd}{    This function should create a new instance of the CNN you created earlier,}
         \PY{l+s+sd}{    load on the weights from the last training epoch, and return this model.}
         \PY{l+s+sd}{    \PYZdq{}\PYZdq{}\PYZdq{}}
             \PY{n}{checkpoint\PYZus{}path}\PY{o}{=}\PY{n}{tf}\PY{o}{.}\PY{n}{train}\PY{o}{.}\PY{n}{latest\PYZus{}checkpoint}\PY{p}{(}
             \PY{l+s+s1}{\PYZsq{}}\PY{l+s+s1}{checkpoints\PYZus{}every\PYZus{}epoch}\PY{l+s+s1}{\PYZsq{}}\PY{p}{,} \PY{n}{latest\PYZus{}filename}\PY{o}{=}\PY{k+kc}{None}
         \PY{p}{)}
             \PY{k}{return} \PY{n}{model}\PY{o}{.}\PY{n}{load\PYZus{}weights}\PY{p}{(}\PY{n}{checkpoint\PYZus{}path}\PY{p}{)}
             
             
             
         \PY{k}{def} \PY{n+nf}{get\PYZus{}model\PYZus{}best\PYZus{}epoch}\PY{p}{(}\PY{n}{model}\PY{p}{)}\PY{p}{:}
             \PY{l+s+sd}{\PYZdq{}\PYZdq{}\PYZdq{}}
         \PY{l+s+sd}{    This function should create a new instance of the CNN you created earlier, load }
         \PY{l+s+sd}{    on the weights leading to the highest validation accuracy, and return this model.}
         \PY{l+s+sd}{    \PYZdq{}\PYZdq{}\PYZdq{}}
             \PY{n}{checkpoint\PYZus{}path}\PY{o}{=}\PY{n}{tf}\PY{o}{.}\PY{n}{train}\PY{o}{.}\PY{n}{latest\PYZus{}checkpoint}\PY{p}{(}
             \PY{l+s+s1}{\PYZsq{}}\PY{l+s+s1}{checkpoints\PYZus{}best\PYZus{}only}\PY{l+s+s1}{\PYZsq{}}\PY{p}{,} \PY{n}{latest\PYZus{}filename}\PY{o}{=}\PY{k+kc}{None}
         \PY{p}{)}
             \PY{k}{return}  \PY{n}{model}\PY{o}{.}\PY{n}{load\PYZus{}weights}\PY{p}{(}\PY{n}{checkpoint\PYZus{}path}\PY{p}{)}
\end{Verbatim}

    \begin{Verbatim}[commandchars=\\\{\}]
{\color{incolor}In [{\color{incolor}57}]:} \PY{c+c1}{\PYZsh{} Run this cell to create two models: one with the weights from the last training}
         \PY{c+c1}{\PYZsh{} epoch, and one with the weights leading to the highest validation (testing) accuracy.}
         \PY{c+c1}{\PYZsh{} Verify that the second has a higher validation (testing) accuarcy.}
         
         \PY{n}{model\PYZus{}last\PYZus{}epoch} \PY{o}{=} \PY{n}{get\PYZus{}model\PYZus{}last\PYZus{}epoch}\PY{p}{(}\PY{n}{get\PYZus{}new\PYZus{}model}\PY{p}{(}\PY{n}{x\PYZus{}train}\PY{p}{[}\PY{l+m+mi}{0}\PY{p}{]}\PY{o}{.}\PY{n}{shape}\PY{p}{)}\PY{p}{)}
         \PY{n}{model\PYZus{}best\PYZus{}epoch} \PY{o}{=} \PY{n}{get\PYZus{}model\PYZus{}best\PYZus{}epoch}\PY{p}{(}\PY{n}{get\PYZus{}new\PYZus{}model}\PY{p}{(}\PY{n}{x\PYZus{}train}\PY{p}{[}\PY{l+m+mi}{0}\PY{p}{]}\PY{o}{.}\PY{n}{shape}\PY{p}{)}\PY{p}{)}
         \PY{n+nb}{print}\PY{p}{(}\PY{l+s+s1}{\PYZsq{}}\PY{l+s+s1}{Model with last epoch weights:}\PY{l+s+s1}{\PYZsq{}}\PY{p}{)}
         \PY{n}{get\PYZus{}test\PYZus{}accuracy}\PY{p}{(}\PY{n}{model\PYZus{}last\PYZus{}epoch}\PY{p}{,} \PY{n}{x\PYZus{}test}\PY{p}{,} \PY{n}{y\PYZus{}test}\PY{p}{)}
         \PY{n+nb}{print}\PY{p}{(}\PY{l+s+s1}{\PYZsq{}}\PY{l+s+s1}{\PYZsq{}}\PY{p}{)}
         \PY{n+nb}{print}\PY{p}{(}\PY{l+s+s1}{\PYZsq{}}\PY{l+s+s1}{Model with best epoch weights:}\PY{l+s+s1}{\PYZsq{}}\PY{p}{)}
         \PY{n}{get\PYZus{}test\PYZus{}accuracy}\PY{p}{(}\PY{n}{model\PYZus{}best\PYZus{}epoch}\PY{p}{,} \PY{n}{x\PYZus{}test}\PY{p}{,} \PY{n}{y\PYZus{}test}\PY{p}{)}
\end{Verbatim}

    \begin{Verbatim}[commandchars=\\\{\}]
WARNING:tensorflow:Unresolved object in checkpoint: (root).optimizer
WARNING:tensorflow:Unresolved object in checkpoint: (root).optimizer.iter
WARNING:tensorflow:Unresolved object in checkpoint: (root).optimizer.beta\_1
WARNING:tensorflow:Unresolved object in checkpoint: (root).optimizer.beta\_2
WARNING:tensorflow:Unresolved object in checkpoint: (root).optimizer.decay
WARNING:tensorflow:Unresolved object in checkpoint: (root).optimizer.learning\_rate
WARNING:tensorflow:Unresolved object in checkpoint: (root).optimizer's state 'm' for (root).layer\_with\_weights-0.kernel
WARNING:tensorflow:Unresolved object in checkpoint: (root).optimizer's state 'm' for (root).layer\_with\_weights-0.bias
WARNING:tensorflow:Unresolved object in checkpoint: (root).optimizer's state 'm' for (root).layer\_with\_weights-1.kernel
WARNING:tensorflow:Unresolved object in checkpoint: (root).optimizer's state 'm' for (root).layer\_with\_weights-1.bias
WARNING:tensorflow:Unresolved object in checkpoint: (root).optimizer's state 'm' for (root).layer\_with\_weights-2.kernel
WARNING:tensorflow:Unresolved object in checkpoint: (root).optimizer's state 'm' for (root).layer\_with\_weights-2.bias
WARNING:tensorflow:Unresolved object in checkpoint: (root).optimizer's state 'm' for (root).layer\_with\_weights-3.kernel
WARNING:tensorflow:Unresolved object in checkpoint: (root).optimizer's state 'm' for (root).layer\_with\_weights-3.bias
WARNING:tensorflow:Unresolved object in checkpoint: (root).optimizer's state 'v' for (root).layer\_with\_weights-0.kernel
WARNING:tensorflow:Unresolved object in checkpoint: (root).optimizer's state 'v' for (root).layer\_with\_weights-0.bias
WARNING:tensorflow:Unresolved object in checkpoint: (root).optimizer's state 'v' for (root).layer\_with\_weights-1.kernel
WARNING:tensorflow:Unresolved object in checkpoint: (root).optimizer's state 'v' for (root).layer\_with\_weights-1.bias
WARNING:tensorflow:Unresolved object in checkpoint: (root).optimizer's state 'v' for (root).layer\_with\_weights-2.kernel
WARNING:tensorflow:Unresolved object in checkpoint: (root).optimizer's state 'v' for (root).layer\_with\_weights-2.bias
WARNING:tensorflow:Unresolved object in checkpoint: (root).optimizer's state 'v' for (root).layer\_with\_weights-3.kernel
WARNING:tensorflow:Unresolved object in checkpoint: (root).optimizer's state 'v' for (root).layer\_with\_weights-3.bias
WARNING:tensorflow:A checkpoint was restored (e.g. tf.train.Checkpoint.restore or tf.keras.Model.load\_weights) but not all checkpointed values were used. See above for specific issues. Use expect\_partial() on the load status object, e.g. tf.train.Checkpoint.restore({\ldots}).expect\_partial(), to silence these warnings, or use assert\_consumed() to make the check explicit. See https://www.tensorflow.org/alpha/guide/checkpoints\#loading\_mechanics for details.

    \end{Verbatim}

    \begin{Verbatim}[commandchars=\\\{\}]

        ---------------------------------------------------------------------------

        AttributeError                            Traceback (most recent call last)

        <ipython-input-57-b6d169507ca4> in <module>
          4 
          5 model\_last\_epoch = get\_model\_last\_epoch(get\_new\_model(x\_train[0].shape))
    ----> 6 model\_best\_epoch = get\_model\_best\_epoch(get\_new\_model(x\_train[0].shape))
          7 print('Model with last epoch weights:')
          8 get\_test\_accuracy(model\_last\_epoch, x\_test, y\_test)


        <ipython-input-55-cdfea231d842> in get\_model\_best\_epoch(model)
         24     'checkpoints\_best\_only', latest\_filename=None
         25 )
    ---> 26     return  model.load\_weights(checkpoint\_path)
    

        /opt/conda/lib/python3.7/site-packages/tensorflow\_core/python/keras/engine/training.py in load\_weights(self, filepath, by\_name)
        179         raise ValueError('Load weights is not yet supported with TPUStrategy '
        180                          'with steps\_per\_run greater than 1.')
    --> 181     return super(Model, self).load\_weights(filepath, by\_name)
        182 
        183   @trackable.no\_automatic\_dependency\_tracking


        /opt/conda/lib/python3.7/site-packages/tensorflow\_core/python/keras/engine/network.py in load\_weights(self, filepath, by\_name)
       1137             format.
       1138     """
    -> 1139     if \_is\_hdf5\_filepath(filepath):
       1140       save\_format = 'h5'
       1141     else:


        /opt/conda/lib/python3.7/site-packages/tensorflow\_core/python/keras/engine/network.py in \_is\_hdf5\_filepath(filepath)
       1447 
       1448 def \_is\_hdf5\_filepath(filepath):
    -> 1449   return (filepath.endswith('.h5') or filepath.endswith('.keras') or
       1450           filepath.endswith('.hdf5'))
       1451 


        AttributeError: 'NoneType' object has no attribute 'endswith'

    \end{Verbatim}

    \hypertarget{load-from-scratch-a-model-trained-on-the-eurosat-dataset.}{%
\paragraph{Load, from scratch, a model trained on the EuroSat
dataset.}\label{load-from-scratch-a-model-trained-on-the-eurosat-dataset.}}

In your workspace, you will find another model trained on the
\texttt{EuroSAT} dataset in \texttt{.h5} format. This model is trained
on a larger subset of the EuroSAT dataset and has a more complex
architecture. The path to the model is \texttt{models/EuroSatNet.h5}.
See how its testing accuracy compares to your model!

    \begin{Verbatim}[commandchars=\\\{\}]
{\color{incolor}In [{\color{incolor}58}]:} \PY{c+c1}{\PYZsh{}\PYZsh{}\PYZsh{}\PYZsh{} GRADED CELL \PYZsh{}\PYZsh{}\PYZsh{}\PYZsh{}}
         
         \PY{c+c1}{\PYZsh{} Complete the following functions. }
         \PY{c+c1}{\PYZsh{} Make sure to not change the function name or arguments.}
         \PY{k+kn}{from} \PY{n+nn}{tensorflow}\PY{n+nn}{.}\PY{n+nn}{keras}\PY{n+nn}{.}\PY{n+nn}{models} \PY{k}{import} \PY{n}{load\PYZus{}model}
         \PY{k}{def} \PY{n+nf}{get\PYZus{}model\PYZus{}eurosatnet}\PY{p}{(}\PY{p}{)}\PY{p}{:}
             \PY{l+s+sd}{\PYZdq{}\PYZdq{}\PYZdq{}}
         \PY{l+s+sd}{    This function should return the pretrained EuroSatNet.h5 model.}
         \PY{l+s+sd}{    \PYZdq{}\PYZdq{}\PYZdq{}}
             \PY{k}{return} \PY{n}{load\PYZus{}model}\PY{p}{(}\PY{l+s+s1}{\PYZsq{}}\PY{l+s+s1}{models/EuroSatNet.h5}\PY{l+s+s1}{\PYZsq{}}\PY{p}{)}
             
\end{Verbatim}

    \begin{Verbatim}[commandchars=\\\{\}]
{\color{incolor}In [{\color{incolor}59}]:} \PY{c+c1}{\PYZsh{} Run this cell to print a summary of the EuroSatNet model, along with its validation accuracy.}
         
         \PY{n}{model\PYZus{}eurosatnet} \PY{o}{=} \PY{n}{get\PYZus{}model\PYZus{}eurosatnet}\PY{p}{(}\PY{p}{)}
         \PY{n}{model\PYZus{}eurosatnet}\PY{o}{.}\PY{n}{summary}\PY{p}{(}\PY{p}{)}
         \PY{n}{get\PYZus{}test\PYZus{}accuracy}\PY{p}{(}\PY{n}{model\PYZus{}eurosatnet}\PY{p}{,} \PY{n}{x\PYZus{}test}\PY{p}{,} \PY{n}{y\PYZus{}test}\PY{p}{)}
\end{Verbatim}

    \begin{Verbatim}[commandchars=\\\{\}]
Model: "sequential\_21"
\_\_\_\_\_\_\_\_\_\_\_\_\_\_\_\_\_\_\_\_\_\_\_\_\_\_\_\_\_\_\_\_\_\_\_\_\_\_\_\_\_\_\_\_\_\_\_\_\_\_\_\_\_\_\_\_\_\_\_\_\_\_\_\_\_
Layer (type)                 Output Shape              Param \#   
=================================================================
conv\_1 (Conv2D)              (None, 64, 64, 16)        448       
\_\_\_\_\_\_\_\_\_\_\_\_\_\_\_\_\_\_\_\_\_\_\_\_\_\_\_\_\_\_\_\_\_\_\_\_\_\_\_\_\_\_\_\_\_\_\_\_\_\_\_\_\_\_\_\_\_\_\_\_\_\_\_\_\_
conv\_2 (Conv2D)              (None, 64, 64, 16)        6416      
\_\_\_\_\_\_\_\_\_\_\_\_\_\_\_\_\_\_\_\_\_\_\_\_\_\_\_\_\_\_\_\_\_\_\_\_\_\_\_\_\_\_\_\_\_\_\_\_\_\_\_\_\_\_\_\_\_\_\_\_\_\_\_\_\_
pool\_1 (MaxPooling2D)        (None, 32, 32, 16)        0         
\_\_\_\_\_\_\_\_\_\_\_\_\_\_\_\_\_\_\_\_\_\_\_\_\_\_\_\_\_\_\_\_\_\_\_\_\_\_\_\_\_\_\_\_\_\_\_\_\_\_\_\_\_\_\_\_\_\_\_\_\_\_\_\_\_
conv\_3 (Conv2D)              (None, 32, 32, 16)        2320      
\_\_\_\_\_\_\_\_\_\_\_\_\_\_\_\_\_\_\_\_\_\_\_\_\_\_\_\_\_\_\_\_\_\_\_\_\_\_\_\_\_\_\_\_\_\_\_\_\_\_\_\_\_\_\_\_\_\_\_\_\_\_\_\_\_
conv\_4 (Conv2D)              (None, 32, 32, 16)        6416      
\_\_\_\_\_\_\_\_\_\_\_\_\_\_\_\_\_\_\_\_\_\_\_\_\_\_\_\_\_\_\_\_\_\_\_\_\_\_\_\_\_\_\_\_\_\_\_\_\_\_\_\_\_\_\_\_\_\_\_\_\_\_\_\_\_
pool\_2 (MaxPooling2D)        (None, 16, 16, 16)        0         
\_\_\_\_\_\_\_\_\_\_\_\_\_\_\_\_\_\_\_\_\_\_\_\_\_\_\_\_\_\_\_\_\_\_\_\_\_\_\_\_\_\_\_\_\_\_\_\_\_\_\_\_\_\_\_\_\_\_\_\_\_\_\_\_\_
conv\_5 (Conv2D)              (None, 16, 16, 16)        2320      
\_\_\_\_\_\_\_\_\_\_\_\_\_\_\_\_\_\_\_\_\_\_\_\_\_\_\_\_\_\_\_\_\_\_\_\_\_\_\_\_\_\_\_\_\_\_\_\_\_\_\_\_\_\_\_\_\_\_\_\_\_\_\_\_\_
conv\_6 (Conv2D)              (None, 16, 16, 16)        6416      
\_\_\_\_\_\_\_\_\_\_\_\_\_\_\_\_\_\_\_\_\_\_\_\_\_\_\_\_\_\_\_\_\_\_\_\_\_\_\_\_\_\_\_\_\_\_\_\_\_\_\_\_\_\_\_\_\_\_\_\_\_\_\_\_\_
pool\_3 (MaxPooling2D)        (None, 8, 8, 16)          0         
\_\_\_\_\_\_\_\_\_\_\_\_\_\_\_\_\_\_\_\_\_\_\_\_\_\_\_\_\_\_\_\_\_\_\_\_\_\_\_\_\_\_\_\_\_\_\_\_\_\_\_\_\_\_\_\_\_\_\_\_\_\_\_\_\_
conv\_7 (Conv2D)              (None, 8, 8, 16)          2320      
\_\_\_\_\_\_\_\_\_\_\_\_\_\_\_\_\_\_\_\_\_\_\_\_\_\_\_\_\_\_\_\_\_\_\_\_\_\_\_\_\_\_\_\_\_\_\_\_\_\_\_\_\_\_\_\_\_\_\_\_\_\_\_\_\_
conv\_8 (Conv2D)              (None, 8, 8, 16)          6416      
\_\_\_\_\_\_\_\_\_\_\_\_\_\_\_\_\_\_\_\_\_\_\_\_\_\_\_\_\_\_\_\_\_\_\_\_\_\_\_\_\_\_\_\_\_\_\_\_\_\_\_\_\_\_\_\_\_\_\_\_\_\_\_\_\_
pool\_4 (MaxPooling2D)        (None, 4, 4, 16)          0         
\_\_\_\_\_\_\_\_\_\_\_\_\_\_\_\_\_\_\_\_\_\_\_\_\_\_\_\_\_\_\_\_\_\_\_\_\_\_\_\_\_\_\_\_\_\_\_\_\_\_\_\_\_\_\_\_\_\_\_\_\_\_\_\_\_
flatten (Flatten)            (None, 256)               0         
\_\_\_\_\_\_\_\_\_\_\_\_\_\_\_\_\_\_\_\_\_\_\_\_\_\_\_\_\_\_\_\_\_\_\_\_\_\_\_\_\_\_\_\_\_\_\_\_\_\_\_\_\_\_\_\_\_\_\_\_\_\_\_\_\_
dense\_1 (Dense)              (None, 32)                8224      
\_\_\_\_\_\_\_\_\_\_\_\_\_\_\_\_\_\_\_\_\_\_\_\_\_\_\_\_\_\_\_\_\_\_\_\_\_\_\_\_\_\_\_\_\_\_\_\_\_\_\_\_\_\_\_\_\_\_\_\_\_\_\_\_\_
dense\_2 (Dense)              (None, 10)                330       
=================================================================
Total params: 41,626
Trainable params: 41,626
Non-trainable params: 0
\_\_\_\_\_\_\_\_\_\_\_\_\_\_\_\_\_\_\_\_\_\_\_\_\_\_\_\_\_\_\_\_\_\_\_\_\_\_\_\_\_\_\_\_\_\_\_\_\_\_\_\_\_\_\_\_\_\_\_\_\_\_\_\_\_
accuracy: 0.810

    \end{Verbatim}

    Congratulations for completing this programming assignment! You're now
ready to move on to the capstone project for this course.


    % Add a bibliography block to the postdoc
    
    
    
    \end{document}
